\documentclass[10pt]{article} % Use the res.cls style, the font size can be changed to 11pt or 12pt here

\usepackage{helvet} % Default font is the helvetica postscript font
\usepackage{hyperref}
\usepackage[a4paper,left=1in,top=1in,right=1in,bottom=1in]{geometry}

\begin{document}

\begin{center}
{\Large \bf Clement Fung - Personal History Statement for UC Berkeley Computer Science PhD Application}
\end{center}

My path to applying to graduate school at UC Berkeley was an unconventional one, with many changes and steps along the way. I first started my undergraduate degree in Systems Design Engineering at the University of Waterloo, a program that emphasizes simulation, product design and human factors engineering. It was not until the summer of my sophomore year in 2013 when I was first exposed to research: at Dr. Paul Boutros' bioinformatics lab at the University of Toronto. It particularly intrigued me to see students working in such a collaborative and supportive academic environment, on problems with unknown solutions. 

From this point onwards, an academic career has always been at the back of my mind. After finishing my undergraduate degree, including 3 internship experiences in the Greater Seattle Area and the Bay Area, I decided to test the waters of graduate school and expand my training in computer science by doing my masters degree in computer science at the University of British Columbia (UBC).

My time at UBC was pivotal in shaping my opinions of graduate school, and choosing my subfield of research in computer science.

When I started my graduate studies at UBC, I took it upon myself to become a champion of diversity and inclusiveness. I was elected as the president of the Computer Science Graduate Student's Association CSGSA) at UBC, this involved overseeing and organizing events for incoming students and serving as a liason between students and faculty within the department.


\end{document}
